\begin{abstract}
Camera imaging sensors - placed adjacent to a surface diffusing the collimated light of a laser - can be used to sense the resulting surface dependent speckle pattern. 
The pattern oscillates if the surface oscillates, leading to the possibility of remote vibrometry. 
While cameras are low cost, easy to use and highly sensitive, they have relatively low frame-rate and thus are a poor choice of sensor for high rate oscillations.
The present work explores the creation of a device to increase the measurement bandwidth while maintaining the sensitivity an off the shelf imaging sensor can provide. 
To this end, a high gain low noise amplifier array has been designed with a bandwidth of 1 \si{\kilo\hertz}. 
The array amplifies the currents of a grid of 3x3 photodiodes which is then low pass filtered and digitized using a high rate ADC.
Tests were conducted to measure the ability of the device to detect laser speckles. Discrete IR lasers, with different beam patterns, and IR dot projectors were evaluated and a detection limit is defined.
We also find that using multiple discrete laser beams illuminating a vibrating piezo disc, and the area around it, increases the signal to noise ratio measureably.
\end{abstract}
